\everymath{\displaystyle}
%\documentclass[pdftex,a4paper]{article}
\documentclass[a4paper]{article}
%%classes: article, report, book, proc, amsproc

%%%%%%%%%%%%%%%%%%%%%%%%
%% Misc
% para acertar os acentos
\usepackage[brazilian]{babel} 
%\usepackage[portuguese]{babel} 
% \usepackage[english]{babel}
% \usepackage[T1]{fontenc}
% \usepackage[latin1]{inputenc}
\usepackage[utf8]{inputenc}
\usepackage{indentfirst}
\usepackage{fullpage}
% \usepackage{graphicx} %See PDF section
\usepackage{multicol}
\setlength{\columnseprule}{0.5pt}
\setlength{\columnsep}{20pt}
%%%%%%%%%%%%%%%%%%%%%%%%
%%%%%%%%%%%%%%%%%%%%%%%%
%% PDF support

\usepackage[pdftex]{color,graphicx}
% %% Hyper-refs
\usepackage[pdftex]{hyperref} % for printing
% \usepackage[pdftex,bookmarks,colorlinks]{hyperref} % for screen

%% \newif\ifPDF
%% \ifx\pdfoutput\undefined\PDFfalse
%% \else\ifnum\pdfoutput > 0\PDFtrue
%%      \else\PDFfalse
%%      \fi
%% \fi

%% \ifPDF
%%   \usepackage[T1]{fontenc}
%%   \usepackage{aeguill}
%%   \usepackage[pdftex]{graphicx,color}
%%   \usepackage[pdftex]{hyperref}
%% \else
%%   \usepackage[T1]{fontenc}
%%   \usepackage[dvips]{graphicx}
%%   \usepackage[dvips]{hyperref}
%% \fi

%%%%%%%%%%%%%%%%%%%%%%%%


%%%%%%%%%%%%%%%%%%%%%%%%
%% Math
\usepackage{amsmath,amsfonts,amssymb}
% para usar R de Real do jeito que o povo gosta
\usepackage{amsfonts} % \mathbb
% para usar as letras frescas como L de Espaco das Transf Lineares
% \usepackage{mathrsfs} % \mathscr

% Oferecer seno e tangente em pt, com os comandos usuais.
\providecommand{\sin}{} \renewcommand{\sin}{\hspace{2pt}\mathrm{sen}}
\providecommand{\tan}{} \renewcommand{\tan}{\hspace{2pt}\mathrm{tg}}

% dt of integrals = \ud t
\newcommand{\ud}{\mathrm{\ d}}
%%%%%%%%%%%%%%%%%%%%%%%%
\date{
\bigskip
Curso: \underline{\hspace{8cm}}\\
\ \\
Turma: \underline{\hspace{1cm}} Série: \underline{\hspace{1cm}} Turno:
\underline{\hspace{1cm}}\\
\ \\
Prof: \underline{\hspace{8cm}}\\
}

\title{Cálculo Diferencial e Integral I - Trabalho 0}

\author{
{\bf Grupo}\\
\ \\
Nome: \underline{\hspace{6cm}} RA: \underline{\hspace{2cm}} Assinatura: \underline{\hspace{4cm}}\\
Nome: \underline{\hspace{6cm}} RA: \underline{\hspace{2cm}} Assinatura: \underline{\hspace{4cm}}\\
Nome: \underline{\hspace{6cm}} RA: \underline{\hspace{2cm}} Assinatura: \underline{\hspace{4cm}}\\
Nome: \underline{\hspace{6cm}} RA: \underline{\hspace{2cm}} Assinatura: \underline{\hspace{4cm}}\\
Nome: \underline{\hspace{6cm}} RA: \underline{\hspace{2cm}} Assinatura: \underline{\hspace{4cm}}\\
Nome: \underline{\hspace{6cm}} RA: \underline{\hspace{2cm}} Assinatura: \underline{\hspace{4cm}}\\
Nome: \underline{\hspace{6cm}} RA: \underline{\hspace{2cm}} Assinatura: \underline{\hspace{4cm}}\\
Nome: \underline{\hspace{6cm}} RA: \underline{\hspace{2cm}} Assinatura: \underline{\hspace{4cm}}\\
Nome: \underline{\hspace{6cm}} RA: \underline{\hspace{2cm}} Assinatura: \underline{\hspace{4cm}}\\
Nome: \underline{\hspace{6cm}} RA: \underline{\hspace{2cm}} Assinatura: \underline{\hspace{4cm}}\\
}


\begin{document}
\maketitle
\newpage

%%%%%%%%%%%%%%%%%%%%%%%%
%% Título e cabeçalho
%\noindent\parbox[c]{.15\textwidth}{\includegraphics[width=.15\textwidth]{logo}}\hfill
\parbox[c]{.825\textwidth}{\raggedright%
  \sffamily {\LARGE

Cálculo Diferencial e Integral I: Trabalho 0

\par\bigskip}
{Prof: Felipe Figueiredo\par}
{\url{http://sites.google.com/site/proffelipefigueiredo}\par}
}

Versão: \verb|20160306|

%%%%%%%%%%%%%%%%%%%%%%%%


%%%%%%%%%%%%%%%%%%%%%%%%

\section{Valor}
O trabalho valerá $0.5$pt {\bf extra}\footnote{para que o(a) aluno(a) receba esta pontuação extra, é necessário que tenha nota positiva na prova. Em caso de dúvida, consulte o link \url{https://sites.google.com/site/proffelipefigueiredo/anhanguera/regras-para-provas}} na P1, desde que {\bf todas} as questões estejam corretas e completas.

\section{Conteúdo}

Questões: seção {\bf 1.7}  do livro texto da disciplina (PLT), fazer os problemas 11 ao 20. Na terceira edição, estes estão na página 40.


\section{Entrega}

O Trabalho 0 deverá ser entregue na próxima aula.

%\url{https://sites.google.com/site/proffelipefigueiredo/anhanguera/}.


\subsection{Observações}

\begin{enumerate}
\item Mesmo se o grupo não conseguir resolver {\bf todos} os problemas, não deixe de tentar! Faça o que conseguir.
\item O grupo deve usar a capa providenciada neste trabalho.
\item A assinatura de cada aluno(a) é obrigatória. Sem assinatura, sem ponto extra.
\item Este trabalho pode ser feito a lápis, ou a caneta. Respostas finais devem estar a caneta, e destacadas.
\end{enumerate}

 


\end{document}
